%%%%%%%%%%%%%%%%%%%%%%%%%%%%%%%%%%%%%%%
% Deedy - One Page Two Column Resume
% LaTeX Template
% Version 1.2 (16/9/2014)
%
% Original author:
% Debarghya Das (http://debarghyadas.com)
%
% Original repository:
% https://github.com/deedydas/Deedy-Resume
%
% IMPORTANT: THIS TEMPLATE NEEDS TO BE COMPILED WITH XeLaTeX
%
% This template uses several fonts not included with Windows/Linux by
% default. If you get compilation errors saying a font is missing, find the line
% on which the font is used and either change it to a font included with your
% operating system or comment the line out to use the default font.
% 
%%%%%%%%%%%%%%%%%%%%%%%%%%%%%%%%%%%%%%
% 
% TODO:
% 1. Integrate biber/bibtex for article citation under publications.
% 2. Figure out a smoother way for the document to flow onto the next page.
% 3. Add styling information for a "Projects/Hacks" section.
% 4. Add location/address information
% 5. Merge OpenFont and MacFonts as a single sty with options.
% 
%%%%%%%%%%%%%%%%%%%%%%%%%%%%%%%%%%%%%%
%
% CHANGELOG:
% v1.1:
% 1. Fixed several compilation bugs with \renewcommand
% 2. Got Open-source fonts (Windows/Linux support)
% 3. Added Last Updated
% 4. Move Title styling into .sty
% 5. Commented .sty file.
%
%%%%%%%%%%%%%%%%%%%%%%%%%%%%%%%%%%%%%%%
%
% Known Issues:
% 1. Overflows onto second page if any column's contents are more than the
% vertical limit
% 2. Hacky space on the first bullet point on the second column.
%
%%%%%%%%%%%%%%%%%%%%%%%%%%%%%%%%%%%%%%


\documentclass[]{deedy-resume-openfont}
\usepackage{fancyhdr}
 
\pagestyle{fancy}
\fancyhf{}



\begin{document}

%%%%%%%%%%%%%%%%%%%%%%%%%%%%%%%%%%%%%%
%
%     LAST UPDATED DATE
%
%%%%%%%%%%%%%%%%%%%%%%%%%%%%%%%%%%%%%%
\lastupdated

%%%%%%%%%%%%%%%%%%%%%%%%%%%%%%%%%%%%%%
%
%     TITLE NAME
%
%%%%%%%%%%%%%%%%%%%%%%%%%%%%%%%%%%%%%%
\namesection{Himanshu}{Dutta}{ \urlstyle{same}\href{https://www.linkedin.com/in/himanshu-dutta-257875179}{Linkedin} | \href{http://github.com/himanshu-dutta/}{GitHub}\\
\href{mailto:meet.himanshu.dutta@gmail.com}{meet.himanshu.dutta@gmail.com} | +91 9828647472 | +91 9407974080
}

%%%%%%%%%%%%%%%%%%%%%%%%%%%%%%%%%%%%%%
%
%     COLUMN ONE
%
%%%%%%%%%%%%%%%%%%%%%%%%%%%%%%%%%%%%%%

\begin{minipage}[t]{0.40\textwidth} 

%%%%%%%%%%%%%%%%%%%%%%%%%%%%%%%%%%%%%%
%     EDUCATION
%%%%%%%%%%%%%%%%%%%%%%%%%%%%%%%%%%%%%%

\section{Education} 

\subsection{KIIT University}
\descript{BTech in Computer Science \& System}
\location{Jun 2018 - Present(Expected 2022) | Bhubaneshwar, India}
\location{ Cum. GPA(3rd Semester): 9.67 / 10.0}
\sectionsep


%%%%%%%%%%%%%%%%%%%%%%%%%%%%%%%%%%%%%%
%     SKILLS
%%%%%%%%%%%%%%%%%%%%%%%%%%%%%%%%%%%%%%

\section{Skills}
\subsection{Technical}
\location{Proficient with:}
Python  \textbullet{} Tensorflow  \textbullet{} C/C++  \textbullet{} Machine Learning  \\
Javascript  \textbullet{} HTML  \textbullet{} CSS \\
Neural Networks  \textbullet{} Data Science  \textbullet{} NLP\\ 
\location{Familiar:}
Django  \textbullet{} MySQL \\
Java  \textbullet{} Reinforcement Learning\\
\sectionsep
\subsection{Soft Skills}
Leadership \textbullet{} Teamwork\\ 
Bilingual(English, Hindi)\\
Problem-Solving \textbullet{} Communication
\sectionsep
\sectionsep
% %%%%%%%%%%%%%%%%%%%%%%%%%%%%%%%%%%%%%%
% %     Experience
% %%%%%%%%%%%%%%%%%%%%%%%%%%%%%%%%%%%%%%

\section{Experience}
\runsubsection{IIT Kanpur}
\descript{| Machine Learning and IoT Internship }
\location{May - July 2019 | Bhubaneshwar, India}
\vspace{\topsep} % Hacky fix for awkward extra vertical space
\begin{tightemize}
\item Underwent 30 days training on Machine Learning and IoT.
\item Learned to integrate different ML models with various IoT devices.
\item Worked on a capstone project based on Natural Language Processing and Timeseries forecasting.
\end{tightemize}
\sectionsep

% %%%%%%%%%%%%%%%%%%%%%%%%%%%%%%%%%%%%%%
% %     Certifications
% %%%%%%%%%%%%%%%%%%%%%%%%%%%%%%%%%%%%%%
\section{Certification} 

\textbullet{}  \href{https://www.coursera.org/verify/E97BDRL3T2L8}{Neural Networks and Deep Learning}\\
\sectionsep
\textbullet{}  \href{https://www.coursera.org/verify/DZ632TAD79WH}{Introduction to TensorFlow for Artificial Intelligence, Machine Learning, and Deep Learning}\\
\sectionsep
\textbullet{}  \href{https://www.coursera.org/verify/TJE3ZUZPNU4K}{Java Programming}\\
\sectionsep
\textbullet{}  \href{https://www.coursera.org/verify/4BECE8N7EVTT}{Programming Foundation with JavaScript, HTML and CSS}\\
\sectionsep

%%%%%%%%%%%%%%%%%%%%%%%%%%%%%%%%%%%%%%
%
%     COLUMN TWO
%
%%%%%%%%%%%%%%%%%%%%%%%%%%%%%%%%%%%%%%

\end{minipage} 
\hfill
\begin{minipage}[t]{0.55\textwidth} 

%%%%%%%%%%%%%%%%%%%%%%%%%%%%%%%%%%%%%%
%     Technical Projects
%%%%%%%%%%%%%%%%%%%%%%%%%%%%%%%%%%%%%%
\section{Technical Projects}

%%%%%%%%%%%%%%%%%%%%%%%%%%%%%%%%%%
\runsubsection{Stock Based Scripts}
\descript{| \href{https://github.com/himanshu-dutta/stock-based-scripts}{GitHub}}
\location{May 2020}
\vspace{\topsep} % Hacky fix for awkward extra vertical space
\begin{tightemize}
\item Wrote various scripts in Python to analyze stock data to base investment decisions on.
\item Created methods to scrap and collect stock data for most of the available stock symbols.
\item Made filters and ranker to assess the quality of stock.
\end{tightemize}
\sectionsep

%%%%%%%%%%%%%%%%%%%%%%%%%%%%%%%%%
\runsubsection{COVID-19 EDA}
\descript{| \href{https://github.com/himanshu-dutta/covid-19}{GitHub}}
\location{April 2020}
\begin{tightemize}
\item Performed exploratory analysis on COVID-19 data made available by Jon Hopkins University.
\item Wrote a notebook to explain SARS model and modelled the pandemic on the same.
\item Employed mathematical models to compare situation in China and India as per their data.
\end{tightemize}
\sectionsep

%%%%%%%%%%%%%%%%%%%%%%%%%%%%%%%%%%
\runsubsection{AI Deabte Master}
\descript{| \href{https://github.com/himanshu-dutta/ai-debate-master}{GitHub}}
\location{April 2020}
\begin{tightemize}
\item Created and deployed a bot made to debate with a human on any given topic.
\item Employed a Doc2Vec model using Gensim library in Python.
\item Used Google CSE and wrote various web-scrapping methods for collection of data.
\end{tightemize}
\sectionsep

%%%%%%%%%%%%%%%%%%%%%%%%%%%%%%%%%%%%%
\runsubsection{Q Learning on custom Chrome T-Rex Environment}
\descript{| \href{https://github.com/himanshu-dutta/chrome_t-rex_environment}{GitHub} }
\location{Oct 2019}
\begin{tightemize}
\item Created a replica of Google Chrome's T-Rex game, ready to be trained using OpenAI's Gym library.
\item Wrote a Q-Learning based script, to train the agent to achieve set goal.
\item For extensive use and learning purpose, the environment is put in an open repository, so as to be used by others to train their agents on it.
\end{tightemize}
\sectionsep

%%%%%%%%%%%%%%%%%%%%%%%%%%%%%%%%%%%%%%
\runsubsection{Effect of News Sentiments on Stock Prediction}
\descript{| \href{https://github.com/himanshu-dutta/sentiment_analysis_on_stock_news}{GitHub} }
\location{June 2019}
\vspace{\topsep} % Hacky fix for awkward extra vertical space
\begin{tightemize}
\item NLP and Time-series forecasting based project.
\item Performed Sentiment Analysis on News related to Stock Market.
\item Compared the forecasting made using LSTM model with and without result of the analysis.
\end{tightemize}
\sectionsep

%%%%%%%%%%%%%%%%%%%%%%%%%%%%%%%%%%%%%%
%     RESEARCH
%%%%%%%%%%%%%%%%%%%%%%%%%%%%%%%%%%%%%%
\section{Research}
\runsubsection{Genetic Algorithm Based Cloud Scheduling}
\descript{| \href{https://github.com/himanshu-dutta/genetic-cloud-task-scheduling}{GitHub}}
\location{Dec 2019}
\begin{tightemize}
\item Employed Genetic Algorithm to schedule tasks on cloud and benchmarked results on \href{https://www.fing.edu.uy/inco/grupos/cecal/hpc/HCSP/index.htm}{HCSP Data} .
\end{tightemize}
\sectionsep

\end{minipage} 
\end{document}  \documentclass[]{article}